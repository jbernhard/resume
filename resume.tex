\documentclass[letterpaper,10pt]{article}

\pagestyle{empty}
\setlength{\parindent}{0ex}
\setlength{\parskip}{.78ex plus .2ex minus .2ex}

\usepackage[margin=.43in]{geometry}

\usepackage{multicol}
\setlength{\columnsep}{.275in}

% compile with xelatex / lualatex
\usepackage{fontspec}
\setmainfont[UprightFont=* Light, BoldFont=* Medium]{Lato}
\newfontfamily{\raleway}{Raleway}
\usepackage[pro]{fontawesome5}

\usepackage{color}
\color[RGB]{32,33,36}

\definecolor{urlcolor}{RGB}{16,86,105}
\usepackage[
  unicode,
  pdftitle={Jonah E. Bernhard resume},
  pdfauthor={Jonah E. Bernhard},
  bookmarks=false,
  colorlinks=true,
  urlcolor=urlcolor
]{hyperref}

\usepackage[explicit]{titlesec}
% \titleformat{command}[shape]{format}{label}{sep}{before-code}[after-code]
\titleformat{\section}{\Large\raleway}{}{0ex}{%
  #1\color[RGB]{121,170,182}\hspace{.75ex}\leavevmode\leaders\hrule height .65ex depth \dimexpr.4pt-.65ex\hfill\kern0pt%
}
\titleformat{\subsection}{\bf}{}{0ex}{#1}
% \titlespacing{command}{left}{before-sep}{after-sep}[right-sep]
\titlespacing{\section}{0ex}{3ex plus 2ex minus 2ex}{.4ex}
\titlespacing{\subsection}{0ex}{.6ex plus .3ex minus .3ex}{0ex minus .8ex}


\begin{document}

\newcommand{\link}[3]{{\color[gray]{.6}\faIcon[solid]{#1}\enspace\href{#2}{#3}}}
\newcommand{\sep}{{\enspace\textperiodcentered\enspace}}

\begin{center}
  {\huge\raleway Jonah E.\ Bernhard} \\[1em]
  Data scientist\sep
  Computational physicist\sep
  Writer, speaker, designer \\[1em]
  \link{globe}{https://jbernhard.xyz}{jbernhard.xyz}\qquad
  \link{envelope}{mailto:jonah.bernhard@gmail.com}{jonah.bernhard@gmail.com}\qquad
  \link{github}{https://github.com/jbernhard}{jbernhard}\qquad
  \link{linkedin}{https://www.linkedin.com/in/jbernhard}{jbernhard}
\end{center}

\smallskip

\begin{multicols}{2}

\newcommand{\subheading}[3]{{\small #1 \hfill #2--#3}}

\section{Education}

\subsection{Duke University}
\subheading{Durham, North Carolina}{August 2011}{April 2018}

Ph.D.\ Physics (2018)

Dissertation: Bayesian parameter estimation for relativistic heavy-ion collisions
(\href{https://arxiv.org/abs/1804.06469}{arXiv:1804.06469 [nucl-th]})

\subsection{Swarthmore College}
\subheading{Swarthmore, Pennsylvania}{September 2007}{May 2011}

B.A.\ Chemical Physics with Highest Honors (2011)

Honors thesis: Lorentz violation in solar-neutrino oscillations


\section{Experience}

\subsection{Data scientist}
\subheading{Lowe's Companies, Inc.}{April 2019}{Present}

Worked on the forecasting team, using statistics and machine learning models to produce forecasts of various business functions, including demand and transportation.
Contributed to improving forecast accuracy.

Supported other ongoing projects on the Data Science team as needed.

\subsection{Graduate research assistant}
\subheading{Duke University (QCD theory group)}{January 2012}{May 2018}

Applied Bayesian parameter estimation to quantify fundamental properties of the quark-gluon plasma created in relativistic heavy-ion collisions.
Achieved the most precise measurements to date, and the first with quantitative uncertainties.

Developed computational models of heavy-ion collisions.
Publicly distributed code on GitHub and wrote documentation.

Used high-performance and high-throughput computational systems to run large-scale simulations of heavy-ion collisions.
Managed and analyzed large scientific datasets.

Published first-author articles in peer-reviewed journals and presented research at international conferences.

\subsection{Teaching assistant \& tutor}
\subheading{Swarthmore College \& Duke University}{2010}{2014}

Taught physics and chemistry recitations.
Privately tutored undergraduate students.

\subsection{Undergraduate research assistant}
\subheading{Swarthmore College}{January 2010}{May 2011}

Developed computational models of neutrino oscillations and calculated the possible effects of Lorentz violation on solar neutrinos.
Derived new theoretical results.


\section{Awards and honors}

Brookhaven National Laboratory Thesis Award (2019)

Brookhaven National Laboratory Merit Award (2018)

Quark Matter Young Scientist Fellowship (2012--2017)

Fellowship to attend ECT* Doctoral Training Program, Trento, Italy (2014)

% Swarthmore College Highest Honors (2011)


\columnbreak


\section{Technical skills}

\subsection{Programming languages}

Python, Cython, C++, C, Bash, Mathematica, JavaScript, PHP, R, Fortran

\subsection{Data analysis, storage \& visualization}

NumPy, SciPy, scikit-learn, pandas; SQL, Hive, Hadoop, HDF5; \\
matplotlib, publication-quality graphics

\subsection{Machine learning}

Neural networks \& deep learning, time series analysis \& forecasting, clustering, linear \& logistic regression, PCA

\subsection{Software engineering}

Agile development, test-driven development, Git
% , building and optimizing computational models of physical systems, creating data analysis tools

\subsection{Bayesian statistics}

Parameter estimation, model calibration, inference, uncertainty quantification, Gaussian processes, MCMC

\subsection{Linux / Unix}

General system administration, HPC \& HTC systems, shells (Bash, Zsh), other standard utilities (awk, sed, etc.)

\subsection{Web development}

HTML, CSS / Sass, Jinja, jQuery, PHP, SQL

\subsection{Document preparation}

Jupyter / IPython, Sphinx, Doxygen, \LaTeX, Mathematica



\section{Publications \& presentations}

\vspace{-.5\parskip}
\emph{%
  \small
  The following is a selection of my most notable works.
  For more information, see my website, \href{https://jbernhard.xyz}{jbernhard.xyz}.
}
\vspace{-.5\parskip}

% \subsection{Ph.D.\ dissertation}

% J.~E.~Bernhard,
% ``Bayesian parameter estimation for relativistic heavy-ion collisions,''
% \href{https://arxiv.org/abs/1804.06469}{arXiv:1804.06469 [nucl-th]}.

\subsection{Journal articles}

J.~E.~Bernhard \textit{et.\ al.},
% ``Bayesian estimation of the specific shear and bulk viscosity of quark–gluon plasma,''
\href{https://www.nature.com/articles/s41567-019-0611-8}{Nature Physics (2019)}.

J.~E.~Bernhard \textit{et.\ al.},
% ``Applying Bayesian parameter estimation to relativistic heavy-ion collisions: Simultaneous characterization of the initial state and quark-gluon plasma medium,''
\href{https://journals.aps.org/prc/abstract/10.1103/PhysRevC.94.024907}{Phys.\ Rev.\ C 94, 024907 (2016)},
\href{https://arxiv.org/abs/1605.03954}{arXiv:1605.03954 [nucl-th]}.

J.~E.~Bernhard \textit{et.\ al.},
% ``Quantifying properties of hot and dense QCD matter through systematic model-to-data comparison,''
\href{https://journals.aps.org/prc/abstract/10.1103/PhysRevC.91.054910}{Phys.\ Rev.\ C 91, 054910 (2015)},
\href{https://arxiv.org/abs/1502.00339}{arXiv:1502.00339 [nucl-th]}.

J.~S.~Moreland, J.~E.~Bernhard, S.~A.~Bass,
% ``Alternative ansatz to wounded nucleon and binary collision scaling in high-energy nuclear collisions,''
\href{https://journals.aps.org/prc/abstract/10.1103/PhysRevC.92.011901}{Phys.\ Rev.\ C 92, 011901 (2015)},
\href{https://arxiv.org/abs/1412.4708}{arXiv:1412.4708 [nucl-th]}.

\subsection{Conference presentations}

Quark Matter: International Conference on Ultra-relativistic Nucleus-Nucleus Collisions, Darmstadt, Germany (2014); Kobe, Japan (2015); Chicago, Illinois (2017)

Bayesian Methods in Nuclear Physics, Institute for Nuclear Theory, University of Washington, Seattle (2016)

Conference on the Intersections of Particle and Nuclear Physics (CIPANP), Vail, Colorado (2015)

RHIC \& AGS Annual Users' Meeting, Brookhaven National Laboratory (2015)


\end{multicols}

\end{document}
