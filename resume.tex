\documentclass[letterpaper,10pt]{article}

\usepackage[utf8]{inputenc}

\pagestyle{empty}
\setlength{\parindent}{0ex}
\setlength{\parskip}{.7ex plus .2ex minus .2ex}

\usepackage[margin=.55in, top=.75in]{geometry}

\usepackage{multicol}
\setlength{\columnsep}{.3in}

% compile with xelatex / lualatex
\usepackage{fontspec}
\setmainfont[UprightFont=* Light, BoldFont=* Medium]{Lato}
\newfontfamily{\raleway}{Raleway}
\usepackage[pro]{fontawesome5}

\usepackage{color}
\color[RGB]{32,33,36}

\definecolor{urlcolor}{RGB}{16,86,105}
\usepackage[
  unicode,
  pdftitle={Jonah E. Bernhard resume},
  pdfauthor={Jonah E. Bernhard},
  bookmarks=false,
  colorlinks=true,
  urlcolor=urlcolor
]{hyperref}

\usepackage[explicit]{titlesec}
% \titleformat{command}[shape]{format}{label}{sep}{before-code}[after-code]
\titleformat{\section}{\Large\raleway}{}{0ex}{%
  #1\color[RGB]{121,170,182}\hspace{.75ex}\leavevmode\leaders\hrule height .65ex depth \dimexpr.4pt-.65ex\hfill\kern0pt%
}
\titleformat{\subsection}{\bf}{}{0ex}{#1}
% \titlespacing{command}{left}{before-sep}{after-sep}[right-sep]
\titlespacing{\section}{0ex}{3ex plus 2ex minus 1.5ex}{.4ex}
\titlespacing{\subsection}{0ex}{.6ex plus .2 ex minus .2ex}{0ex minus .2ex}


\begin{document}

\newcommand{\link}[3]{{\color[gray]{.6}\faIcon[solid]{#1}\enspace\href{#2}{#3}}}
\newcommand{\sep}{{\enspace\textperiodcentered\enspace}}

\begin{center}
  {\huge\raleway Jonah E.\ Bernhard} \\[1em]
  Computational physicist\sep
  Data scientist\sep
  Optimal solution finder\sep
  Writer, speaker, designer\\[1em]
  % \link{globe}{https://jbernhard.xyz}{jbernhard.xyz}\qquad
  \link{envelope}{mailto:jonah.bernhard@gmail.com}{jonah.bernhard@gmail.com}\qquad
  \link{github}{https://github.com/jbernhard}{jbernhard}\qquad
  \link{linkedin}{https://www.linkedin.com/in/jbernhard}{jbernhard}
\end{center}

\smallskip

\begin{multicols}{2}

\section{Education}

\subsection{Duke University}
\vspace{-\parskip}
{\small Durham, North Carolina \hfill August 2011--March 2018}

Ph.D.\ Physics (2018)

Dissertation: Bayesian parameter estimation for relativistic heavy-ion collisions
(\href{https://arxiv.org/abs/1804.06469}{arXiv:1804.06469 [nucl-th]})

\subsection{Swarthmore College}
\vspace{-\parskip}
{\small Swarthmore, Pennsylvania \hfill September 2007--May 2011}

B.A.\ Chemical Physics with Highest Honors (2011)

Honors thesis: Lorentz violation in solar-neutrino oscillations


\section{Experience}

\subsection{Graduate research assistant}
\vspace{-\parskip}
{\small Duke University (QCD theory group) \hfill January 2012--April 2018}

Applied Bayesian parameter estimation methods to quantify fundamental properties of the quark-gluon plasma created in relativistic heavy-ion collisions.
Achieved the most precise measurements to date, and the first with quantitative uncertainties.

Developed computational models of heavy-ion collisions.
Publicly distributed code on Github and wrote documentation.

Used high-performance and high-throughput computational systems to run large-scale simulations of heavy-ion collisions.
Managed and analyzed large scientific datasets.

Published first-author articles in peer-reviewed journals and presented research at international conferences.

\subsection{Undergraduate research assistant}
\vspace{-\parskip}
{\small Swarthmore College \hfill January 2010--May 2011}

Modeled and quantified the possible effects of Lorentz violation on solar neutrinos.

Developed computational models of neutrino oscillations and derived new theoretical results.

\subsection{Summer research fellow}
\vspace{-\parskip}
{\small Swarthmore College \hfill June--August 2009}

Performed experimental studies of the excited states of nitric oxide gas.

Operated lasers and other chemistry lab equipment.
Analyzed spectroscopic data.


\section{Awards and honors}

Brookhaven National Laboratory Merit Award (2018)

Quark Matter Young Scientist Fellowship (2012--2017)

ECT* Doctoral Training Program Fellowship (2014)

Swarthmore College Highest Honors (2011)


\columnbreak


\section{Technical skills}

\subsection{Programming languages}

Python, Cython, C++, C, Bash, Mathematica, JavaScript, PHP, R, Fortran

\subsection{Bayesian methods}

Parameter estimation, model calibration, uncertainty quantification, Gaussian processes, MCMC / Monte Carlo methods

\subsection{Data analysis and storage}

numpy, scipy, scikit-learn, pandas;
HDF5, SQL

\subsection{Software engineering}

Building computational models of physical systems, creating data analysis tools, Git, test-driven development

\subsection{High-performance computing}

MPI, OpenMP, SLURM, HTCondor

\subsection{Linux / Unix}

General system administration, shells (Bash, Zsh), other standard utilities (awk, sed, etc)

\subsection{Web development}

HTML, CSS / Sass, Jinja, jQuery, PHP, SQL

\subsection{Document preparation}

Jupyter / IPython, Sphinx, Doxygen, \LaTeX, Mathematica



\section{Publications}

\emph{%
  The following is a selection of my most notable works.
  For a complete list, see my website, \href{https://jbernhard.xyz}{jbernhard.xyz}.
}

\subsection{Ph.D.\ dissertation}

J.~E.~Bernhard,
``Bayesian parameter estimation for relativistic heavy-ion collisions,''
\href{https://arxiv.org/abs/1804.06469}{arXiv:1804.06469 [nucl-th]}.

\subsection{Journal articles}

J.~E.~Bernhard \textit{et.\ al.},
``Applying Bayesian parameter estimation to relativistic heavy-ion collisions: Simultaneous characterization of the initial state and quark-gluon plasma medium,''
\href{https://journals.aps.org/prc/abstract/10.1103/PhysRevC.94.024907}{Phys.\ Rev.\ C 94, 024907 (2016)},
\href{https://arxiv.org/abs/1605.03954}{arXiv:\allowbreak 1605.03954 [nucl-th]}.

J.~E.~Bernhard \textit{et.\ al.},
``Quantifying properties of hot and dense QCD matter through systematic model-to-data comparison,''
\href{https://journals.aps.org/prc/abstract/10.1103/PhysRevC.91.054910}{Phys.\ Rev.\ C 91, 054910 (2015)},
\href{https://arxiv.org/abs/1502.00339}{arXiv:1502.00339 [nucl-th]}.

% J.~S.~Moreland, J.~E.~Bernhard, S.~A.~Bass,
% ``Alternative ansatz to wounded nucleon and binary collision scaling in high-energy nuclear collisions,''
% \href{https://journals.aps.org/prc/abstract/10.1103/PhysRevC.92.011901}{Phys.\ Rev.\ C 92, 011901 (2015)},
% \href{https://arxiv.org/abs/1412.4708}{arXiv:1412.4708 [nucl-th]}.

\subsection{Conference proceedings}

J.~E.~Bernhard \textit{et.\ al.},
``Characterization of the initial state and QGP medium from a combined Bayesian analysis of LHC data at 2.76 and 5.02 TeV,''
\href{https://www.sciencedirect.com/science/article/pii/S0375947417301549}{Nucl.\ Phys.\ A 967, 293 (2017)},
\href{https://arxiv.org/abs/1704.04462}{arXiv:1704.04462 [nucl-th]},
\href{https://indico.cern.ch/event/433345/contributions/2358284}{presented at Quark Matter 2017}.

\end{multicols}

\end{document}
